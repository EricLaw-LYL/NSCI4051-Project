\documentclass[11pt,a4paper]{article}
\input{myPreliminary}

\newcounter{magicrownumbers}
\newcommand\rownumber{\stepcounter{magicrownumbers}\arabic{magicrownumbers}}

\usepackage[table]{xcolor}
%------------------------------------------------------------------------------

\begin{document}
    
    % Page 0 for names and table of contents
    \thispagestyle{empty}
    \pagenumbering{gobble} 
    \title{\textsc{NSCI 4051} -- Workshop on Data Sciences}
    \author{
        LAW Yiu Leung Eric (SID: \texttt{1155149315})
    }
    \date{\today}
    \maketitle

    \begin{center}
        \textbf{Topic: Predict Student Performance from Game Play} \\
        Instructor: Dr. Edmond Chan
    \end{center}
    
    \pagestyle{plain} 
    % \pagenumbering{roman}
    
    \tableofcontents
    \listoffigures
    \listoftables
    
    \newpage
    
    % Section 1
    \pagenumbering{arabic}
    \pagestyle{fancy}
    \setcounter{page}{1}
    
    % Emotion Detection
    \section{Introduction}
    Game-based learning is a method of education that has seen growing popularity in recent years. It involves using gaming elements and mechanics to teach academic concepts and skills, making learning a more interactive and entertaining experience for students. This approach to education has been shown to be effective in engaging students and improving their academic outcomes. \\
    \\
    The lack of knowledge tracing in game-based learning platforms is a missed opportunity to provide individualized support to students, which can ultimately improve their academic outcomes. Therefore, there is a need for increased focus on incorporating knowledge tracing techniques in educational games to support students' learning and development. \\
    \\
    The project is try to make advancement of knowledge-tracing methods for game-based learning. This will benefit the developers of educational games by providing them with valuable insights on how to create more effective learning experiences for their students. Ultimately, this work aims to enhance the quality of education through the use of game-based learning and promote better academic outcomes for students.
    
    
    \section{Bibliography}
    \bibliographystyle{unsrt}
    % \bibliographystyle{plain} % We choose the "plain" reference style
    \bibliography{refs} % Entries are in the refs.bib file
    
\end{document}